\documentclass[11pt]{article}
\usepackage[a4paper,margin=1in]{geometry}
\usepackage{graphicx}
\usepackage{booktabs}
\usepackage{hyperref}
\usepackage{longtable}
\usepackage{float}
\usepackage{amsmath}
\usepackage{enumitem}

\title{Understanding the AI Development Workflow\\\large Hospital Readmission Case Study}
\author{Team Placeholder}
\date{\today}

\begin{document}
\maketitle

\begin{abstract}
This report demonstrates the AI Development Workflow for predicting 30-day patient readmissions. It covers problem framing, data strategy, model development, evaluation, deployment, monitoring, and ethical considerations while reflecting on challenges and future improvements.
\end{abstract}

\section{Problem Definition}
\subsection{Hypothetical Scenario}
\begin{itemize}[leftmargin=*]
    \item \textbf{Use Case:} Predict student dropout risk in an online bootcamp.
    \item \textbf{Objectives:} (1) Achieve AUROC $\geq 0.85$, (2) surface top risk factors, (3) prioritize interventions for high-risk cohorts.
    \item \textbf{Stakeholders:} Program directors, student success coaches.
    \item \textbf{KPI:} Area Under ROC Curve (AUROC) on holdout data.
\end{itemize}

\subsection{Case Study Scenario}
\begin{itemize}[leftmargin=*]
    \item \textbf{Problem:} Forecast 30-day hospital readmission risk.
    \item \textbf{Objectives:} Reduce readmission penalties, enable proactive care, deliver explainable insights.
    \item \textbf{Stakeholders:} Chief Medical Officer, clinicians, discharge planners, compliance officers, patients.
\end{itemize}

\section{Data Strategy}
\subsection{Data Sources}
\begin{enumerate}[leftmargin=*]
    \item Electronic Health Records (EHRs) with diagnoses, vitals, and lab results.
    \item Insurance claims capturing prior admissions and medication adherence.
    \item Social determinants datasets (income, transportation).
    \item Clinical notes (discharge summaries).
\end{enumerate}

\subsection{Ethical Considerations}
\begin{itemize}[leftmargin=*]
    \item Representation bias due to under-sampled populations.
    \item Measurement bias from uneven data capture.
    \item Mitigation via fairness audits, reweighting, community oversight.
\end{itemize}

\subsection{Preprocessing Pipeline}
\begin{enumerate}[leftmargin=*]
    \item Data cleaning and imputation (median for numerics, mode for categoricals).
    \item Scaling numeric features; one-hot encoding categorical variables.
    \item Feature engineering: comorbidity index, medication possession ratio, socio-economic scores.
    \item Text featurization with TF-IDF or clinical language models.
    \item SMOTE oversampling for class imbalance.
\end{enumerate}

\section{Model Development}
\subsection{Model Choice}
Gradient boosting (LightGBM) is selected for its strong performance on tabular data, ability to handle missing values, and compatibility with SHAP explanations. Logistic regression serves as a baseline; deep learning models are considered but deprioritized due to interpretability and compute constraints.

\subsection{Data Splitting}
Temporal split: 70\% train, 15\% validation, 15\% test to avoid leakage. Stratification ensures balanced class representation.

\subsection{Hyperparameter Tuning}
\begin{itemize}[leftmargin=*]
    \item \texttt{num\_leaves} -- controls tree complexity.
    \item \texttt{learning\_rate} -- balances convergence speed and overfitting.
    \item \texttt{min\_child\_samples}, \texttt{subsample}, \texttt{colsample\_bytree} -- regularization and diversity.
\end{itemize}

\section{Evaluation}
\subsection{Metrics}
AUROC, Average Precision, F1, Precision, Recall, and calibration curves. Threshold-based confusion matrix to support clinical decision-making.

\subsection{Hypothetical Confusion Matrix}
\begin{center}
\begin{tabular}{lcc}
\toprule
 & \textbf{Predicted Readmit} & \textbf{Predicted Not Readmit} \\
\midrule
\textbf{Actual Readmit} & 120 (TP) & 30 (FN) \\
\textbf{Actual Not Readmit} & 50 (FP) & 300 (TN) \\
\bottomrule
\end{tabular}
\end{center}
Precision = 0.706, Recall = 0.800, F1 = 0.750, AUROC = 0.87.

\subsection{Concept Drift}
Concept drift arises when patient populations or clinical practices change over time. We monitor PSI monthly; trigger retraining if PSI $\geq 0.2$ or validation AUROC decreases by 0.02.

\section{Deployment}
\subsection{Integration Steps}
\begin{enumerate}[leftmargin=*]
    \item Containerize preprocessing and model service with FastAPI.
    \item Deploy to hospital Kubernetes cluster; expose gRPC/REST endpoints.
    \item Batch score daily discharges; push alerts via FHIR API to EHR dashboard.
    \item Log predictions, SHAP explanations, and clinician feedback.
\end{enumerate}

\subsection{Regulatory Compliance}
Apply HIPAA safeguards, maintain audit trails, conduct privacy impact assessments, and document model governance (versioning, change logs).

\section{Optimization}
Mitigate overfitting via early stopping, regularization (\texttt{reg\_alpha}, \texttt{reg\_lambda}), and nested cross-validation. Perform feature importance pruning and monitor learning curves.

\section{Critical Thinking}
\subsection{Ethics \& Bias}
Biased training data can misclassify vulnerable patients, leading to inadequate care. Mitigation includes fairness constraints, stratified evaluation, and clinician review.

\subsection{Trade-offs}
Balancing interpretability and accuracy is crucial; tree-based models with SHAP explanations offer transparency with competitive performance. Limited compute resources favor gradient boosting over deep neural networks.

\section{Reflection}
Most challenging task: aligning privacy, compliance, and data access. With more time, we would build automated MLOps pipelines, expand SDOH coverage, and run clinician usability studies.

\section{Workflow Diagram}
The AI development workflow follows the CRISP-DM inspired loop:
\begin{enumerate}[leftmargin=*]
    \item Problem Definition
    \item Data Collection \& Governance
    \item Preprocessing \& Feature Engineering
    \item Model Development
    \item Evaluation \& Validation
    \item Deployment \& Integration
    \item Monitoring \& Maintenance with feedback into earlier stages
\end{enumerate}

\section{References}
\begin{enumerate}[leftmargin=*]
    \item CRISP-DM Consortium. ``CRISP-DM 1.0: Step-by-step data mining guide.'' 2000.
    \item Ribeiro, M. et al. ``Why Should I Trust You?'' KDD 2016.
    \item Rajkomar, A. et al. ``Machine Learning in Medicine.'' NEJM 2019.
    \item U.S. Department of Health and Human Services. ``Summary of the HIPAA Privacy Rule.'' 2022.
    \item Wiens, J. et al. ``Do no harm: a roadmap for responsible machine learning for health care.'' Nature Medicine, 2019.
\end{enumerate}

\end{document}

